\chapter{Introduction}

\section{Motivation}

In recent years, there has been an explosion of cloud computing software. After Google published their paper on MapReduce \citep{dean2010mapreduce}, many new open-source systems for distributed computation have emerged, most notably Apache Hadoop \citep{ApacheHadoop} for batch processing and Apache Storm \citep{ApacheStorm} for real-time data stream processing.

These systems split the work that needs to be carried out and distribute it across nodes of a commodity hardware cluster. Commercial companies and researchers have been able to utilise these systems and create distributed systems which can accomplish things that would not be otherwise possible \cite{Solovey, Bifet:2010:MMO:1756006.1859903}. This has mostly been allowed by low price and good scale-out properties of commodity hardware.

At the same time, chip makers have been increasing the number of cores in processors and now we are at a point where servers with 10-core processors are considered standard. Moreover, most high-end servers have multiple processor sockets thus increasing the parallelisation possible with a single machine even further.

The number of cores is increasing but the price of highly parallel machines has gone down. In 2008, a typical Hadoop node had two dual-core processors and 4 GB of random access memory (RAM) \citep{Kumar:2013:HSD:2536274.2536314}. Nowadays, a server with two eight-core processors and 256 GB of RAM can be purchased for roughly \$10,000 USD \citep{Kumar:2013:HSD:2536274.2536314}. Hence a single server today might have better processing power than a small cluster from a few years ago. If this trend continues there will be processors with even more cores in the near future with higher processing power than most clusters today.

Moreover, tiled processors have emerged as competitors to traditional processors in throughput-based computations \cite{Tilera}. These processors use a large number of tiles connected by an on-chip network and even though the single-thread performance of each tile is lower than the performance of a conventional core, the increased parallelism yields higher throughput \cite{DBLP:conf/isca/Lotfi-KamranGFVKPAJIOF12}.

Seeing these trends, we believe there is a place for a real-time data stream processing system running on a single multi-core machine.

It is generally believed that writing parallel software is hard. The traditional techniques of message passing and shared memory require the programmer to manage concurrency at a fairly low level. Furthermore, Apache Storm has become the \textit{de facto} tool used in stream processing on a cluster and according to their ``Powered By" page there are tens of companies already using Storm to process their real-time data streams \cite{PoweredBy}. We think that Storm's popularity and easy to use application programming interface (API) makes it the ideal candidate for porting to multi-core.

\section{Project Contributions}

The main idea of this project is to take the existing Apache Storm project and port it to multi-core. This is implemented in Storm-MC - a library with an API compatible with Apache Storm. This compatibility enables programmers to take an existing application written with Apache Storm in mind and run it on a single multi-core server using Storm-MC. This way, they can avoid network latency and enjoy the substantial performance improvements of a shared-memory environment.

\todo{Add borrowing ideas from distributed systems to multi-core.}

Through a series of benchmarks we show that with its simpler design, Storm-MC offers substantial improvement in throughput for data stream processing applications over Apache Storm running in local mode.

\section{Structure of the Report}

The remainder of the report is structured as follows:

\begin{itemize}
	\item Chapter 2 presents an overview of related literature and gives background on data stream processing and multi-core architectures.
	\item Chapter 3 explains the concepts used in Apache Storm as well as the architecture of a Storm cluster.
	\item Chapter 4 describes how Apache Storm was ported to multi-core and explains the design of Storm-MC.
	\item Chapter 5 presents results of benchmarking Storm-MC against Apache Storm running in local mode.
	\item Chapter 6 summarises this report and presents considerations for future work.
\end{itemize}
