% to choose your degree
% please un-comment just one of the following
\documentclass[bsc,logo,frontabs,twoside,singlespacing,normalheadings,parskip]{infthesis}\usepackage[]{graphicx}\usepackage[]{color}
%% maxwidth is the original width if it is less than linewidth
%% otherwise use linewidth (to make sure the graphics do not exceed the margin)
\makeatletter
\def\maxwidth{ %
  \ifdim\Gin@nat@width>\linewidth
    \linewidth
  \else
    \Gin@nat@width
  \fi
}
\makeatother

\definecolor{fgcolor}{rgb}{0.345, 0.345, 0.345}
\newcommand{\hlnum}[1]{\textcolor[rgb]{0.686,0.059,0.569}{#1}}%
\newcommand{\hlstr}[1]{\textcolor[rgb]{0.192,0.494,0.8}{#1}}%
\newcommand{\hlcom}[1]{\textcolor[rgb]{0.678,0.584,0.686}{\textit{#1}}}%
\newcommand{\hlopt}[1]{\textcolor[rgb]{0,0,0}{#1}}%
\newcommand{\hlstd}[1]{\textcolor[rgb]{0.345,0.345,0.345}{#1}}%
\newcommand{\hlkwa}[1]{\textcolor[rgb]{0.161,0.373,0.58}{\textbf{#1}}}%
\newcommand{\hlkwb}[1]{\textcolor[rgb]{0.69,0.353,0.396}{#1}}%
\newcommand{\hlkwc}[1]{\textcolor[rgb]{0.333,0.667,0.333}{#1}}%
\newcommand{\hlkwd}[1]{\textcolor[rgb]{0.737,0.353,0.396}{\textbf{#1}}}%

\usepackage{framed}
\makeatletter
\newenvironment{kframe}{%
 \def\at@end@of@kframe{}%
 \ifinner\ifhmode%
  \def\at@end@of@kframe{\end{minipage}}%
  \begin{minipage}{\columnwidth}%
 \fi\fi%
 \def\FrameCommand##1{\hskip\@totalleftmargin \hskip-\fboxsep
 \colorbox{shadecolor}{##1}\hskip-\fboxsep
     % There is no \\@totalrightmargin, so:
     \hskip-\linewidth \hskip-\@totalleftmargin \hskip\columnwidth}%
 \MakeFramed {\advance\hsize-\width
   \@totalleftmargin\z@ \linewidth\hsize
   \@setminipage}}%
 {\par\unskip\endMakeFramed%
 \at@end@of@kframe}
\makeatother

\definecolor{shadecolor}{rgb}{.97, .97, .97}
\definecolor{messagecolor}{rgb}{0, 0, 0}
\definecolor{warningcolor}{rgb}{1, 0, 1}
\definecolor{errorcolor}{rgb}{1, 0, 0}
\newenvironment{knitrout}{}{} % an empty environment to be redefined in TeX

\usepackage{alltt}     % for BSc, BEng etc.
% \documentclass[minf,frontabs,twoside,singlespacing,parskip]{infthesis}  % for MInf

% APPENDIX
\usepackage[page]{appendix}

% FONT COMMANDS
\usepackage{fontspec}
\setmainfont[Mapping=tex-text,Numbers=OldStyle]{fbb}
\setsansfont[Mapping=tex-text,Numbers=OldStyle,Scale=MatchLowercase]{Gill Sans}
\setmonofont[Mapping=tex-text,Scale=MatchLowercase]{Inconsolata}


% DISPLAY TODOS
%\usepackage[disable]{todonotes}
\usepackage[draft,bordercolor=white,backgroundcolor=yellow!60,linecolor=black,colorinlistoftodos]{todonotes}

% MAKE SURE TODOS ARE INLINE
\let\Oldtodo\todo
\renewcommand{\todo}[1]{\Oldtodo[inline]{#1}}

% INCLUDE CODE FILES
\usepackage{minted}

% ALLOW TWO FIGURES NEXT TO EACH OTHER
\usepackage{subcaption}


% ENSURE CHAPTERS WORK IN PDF VIEWERS
\usepackage[]{hyperref}
\hypersetup{
    pdftitle={Storm on Multi-core},
    pdfauthor={Mark Nemec},
    %pdfsubject={Your subject here},
    %pdfkeywords={keyword1, keyword2},
    bookmarksnumbered=true,
    bookmarksopen=true,
    bookmarksopenlevel=1,
    %colorlinks=true,
    pdfstartview=Fit,
    pdfpagemode=UseOutlines,
    pdfpagelayout=TwoPageRight
}
\IfFileExists{upquote.sty}{\usepackage{upquote}}{}
\begin{document}

\title{Storm on Multi-core}
\author{Mark Nemec}
\course{Computer Science}
\project{4th Year Project Report}

\date{\today}

\abstract{This is the abstract.}

\maketitle

\section*{Acknowledgements}
Acknowledgements go here.

\tableofcontents
\listoflistings

\pagenumbering{arabic}

\chapter{Introduction}

\section{Motivation}

In recent years, there has been an explosion of cloud computing software. After Google published their paper on MapReduce \citep{dean2010mapreduce}, many new open-source frameworks for distributed computation have emerged, most notably Apache Hadoop \citep{ApacheHadoop} for batch processing and Apache Storm \citep{ApacheStorm} for real-time data stream processing.

The main idea of these frameworks is to split the work that needs to be carried out and distribute it across nodes of a cluster. Commercial companies and researchers have been able to utilise these frameworks and create distributed systems which can accomplish things that would not be otherwise possible. This has mostly been allowed by the low price and good horizontal scaling properties of commodity hardware.

%\todo{Quote that they are standard.}

At the same time, chip makers have been increasing the number of cores in processors and now we are at a point where servers with 10-core processors are standard. Moreover, most high-end servers support multiple processor sockets thus furthering the parallelisation possible with a single machine even more.

The price of multi-core servers has been going down as well. In 2008, a typical Hadoop node had two dual-core processors and 4 GB of random access memory (RAM). Nowadays, a server with two eight-core processors and 256 GB of RAM can be purchased for roughly \$10,000 USD \citep{Kumar:2013:HSD:2536274.2536314}. Hence a single server today might have better processing power than a small cluster from a few years ago \citep{Kumar:2013:HSD:2536274.2536314}.

Even though the cost of a commodity hardware cluster might be lower than the price of a single computer with equal power there are certain limitations. These limitations are further explored in section \ref{sec:multicore}.

\section{Main Idea}

The main idea of this report is to take the existing Apache Storm project and port it to multi-core. This is implemented in Storm-MC - a library with an API compatible with Apache Storm. This allows programmers to take an existing application written with Apache Storm in mind and run it on a multi-core server. This way, we can avoid network latency and enjoy the significant performance improvements of a shared-memory environment.

\section{Structure of the Report}

The remainder of the report is structured as follows:

\begin{itemize}
	\item \textbf{Chapter 2} presents an overview of related literature and gives background on data stream processing and multi-core architectures.
	\item \textbf{Chapter 3} explains the concepts used in Apache Storm as well as the architecture of a Storm cluster.
	\item \textbf{Chapter 4} describes how Apache Storm was ported over to Storm-MC.
	\item \textbf{Chapter 5} gives an overview the implementation details of Storm-MC.
	\item \textbf{Chapter 6} discusses the evaluation results of Storm-MC.
	\item \textbf{Chapter 7} presents the conclusion of this report.
\end{itemize}


\include{prior_work}

\chapter{Background on Storm}

In this chapter we give background information on Storm. This information is necessary to understand the design of Storm-MC. We give a quick overview of Apache Storm (\ref{sec:storm_overview}), explain the concepts used in Storm (\ref{sec:concepts}), show an example Storm program (\ref{sec:example_topology}), give details about the underlying architecture of Storm (\ref{sec:storm_arch}), and finally describe the serialisation used by Storm (\ref{sec:serialisation}).

\section{Storm Overview}
\label{sec:storm_overview}

Apache Storm was developed in a mix of Java and Clojure. As mentioned by the author of Storm in \cite{Marz_2014}, writing the Storm interfaces in Java ensured large potential user-base while writing the implementation in Clojure increased productivity.

To ensure API compatibility with Storm, Storm-MC was developed using the same set of languages. This allowed for code reuse and not having to re-implement functionality already present in Storm. Hence, in the following sections we describe Storm in greater detail in hope that this will later clarify design choices made for Storm-MC.

\section{Storm Concepts}
\label{sec:concepts}

\subsection{Core Concepts}

There are several core concepts used by Storm and hence by extension Storm-MC as well. These concepts are put together to form a simple API that allows the programmer to break down a computation into separate components and define how these components interact with each other. The three core concepts of Storm are:

\begin{description}
  \item[Spout] \hfill \\
  A spout is a component that represents the source of a data-stream. Typically, a spout reads from a message broker such as RabbitMQ \cite{RabbitMQ} or Apache Kafka but can also generate its own stream or read from somewhere like the Twitter streaming API \citep{TwitterStreaming}.
  \item[Bolt] \hfill \\
  A bolt is a component that transforms tuples from its input data stream and emits them to its output data stream. A bolt can perform a range of functions e.g. filter out tuples based on some criteria or perform a join of two different input streams.
  \item[Topology] \hfill \\
  The programmer connects spouts and bolts in a directed graph called topology which describes how the components interact with each other. The topology is then submitted to Storm for execution.
\end{description}

\subsection{Additional Concepts}

There are several additional concepts which describe how components of a topology interact:

\begin{description}
  \item[Stream] \hfill \\
  A stream is defined as an unbounded sequence of tuples. Streams can be thought of as edges of a topology connecting bolts and spouts (vertices).
  \item[Tuple] \hfill \\
  A tuple wraps named fields and their values. The values of the fields can be of different types. When a component emits a tuple to a stream it sends that tuple to every bolt subscribed to the stream.
  \item[Stream Grouping] \hfill \\
  Every bolt needs to have a type of stream grouping associated with it. This grouping decides the means of distributing the tuples coming from a bolt's input stream among task instances of the bolt. Following are the possible types of stream grouping:
  \begin{description}
  	\item[Shuffle] Randomly partition the tuples among all the bolt tasks.
  	\item[Fields] Hash on a subset of the tuple fields. All tuples with same values of those fields will go to same bolt task.
  	\item[All] Replicate the entire stream to all the bolt tasks.
  	\item[Direct] The producer of the tuple decides which task of the bolt will receive this tuple.
  	\item[Global] Send the entire stream to a single bolt task.
  	\item[Local or Shuffle] Prefer sending to executors in the same worker process, if that is not possible use same strategy as Shuffle.
  \end{description}
\end{description}

Users are also able to specify their own custom grouping by implementing the \texttt{CustomStreamGrouping} interface.

All the components of a Storm topology execute in parallel. The user can specify how much parallelism he wants associated with every component and Storm spawns the necessary number of threads. This is done through a configuration file, defined in YAML, which is submitted along with the topology.

There are two additional bolts running for every topology:

\begin{description}
	\item[Acker] \hfill \\
	The Acker bolt guarantees fault tolerance for the topology. It tracks every tuple that was produced and ensures that the tuple has been acknowledged by every bolt of the stream.
	\item[System Bolt] \hfill \\
	The System bolt is useful in two ways:
	\begin{description}
		\item[Metrics] System bolt collects metrics on the local Java Virtual Machine (JVM). Other components can subscribe to these metrics and receive their values at  regular intervals.
		\item[Ticks] Components of a topology can subscribe to receive tick tuples in regular intervals. These tuples can be used to trigger some event of a component.
	\end{description}
\end{description}

\section{Example Topology}
\label{sec:example_topology}

\begin{figure}[!htb]
	\centering
	\includegraphics[scale=0.475]{pdf/wordcount_topology.pdf}
	\caption{WordCount topology.}
	\label{fig:wordcount_topology}
\end{figure}

A classic example used to explain Storm topologies is the WordCount topology. In this topology, there is a spout generating random sentences, a bolt splitting the sentences on white space, and a bolt counting occurrences of every word. Figure \ref{fig:wordcount_topology} shows how we could represent this topology graphically.

This may seem as a simplistic example but it is useful when demonstrating how easy it is to implement a working topology using the Storm API.

Listing \ref{listing:wordcount_topology} shows how the topology is put together in Storm to form a graph of components. Storm uses the Builder design pattern \citep{gamma1994design} to build the topology which is then submitted to an emulated cluster for execution. The last argument to the~\texttt{setBolt}/\texttt{setSpout} method is the number of parallel tasks we want Storm to execute for the respective component. For implementation of the spout and the bolts used in this topology, refer to Appendix \ref{ch:listings}.

\begin{listing}[!htb]
\inputminted{java}{code/WordCountTopology.java}
\caption{WordCountTopology.java}
\label{listing:wordcount_topology}
\end{listing}

\section{Storm Architecture}
\label{sec:storm_arch}

\begin{figure}[!htb]
	\centering
	\includegraphics[scale=0.4]{pdf/storm_arch.pdf}
	\caption{Apache Storm architecture.}
	\label{fig:storm_arch}
\end{figure}

%\todo{Maybe highlight similarities to Hadoop}

A Storm cluster adopts the Master-Worker pattern. To set up a Storm topology, the user launches daemon processes on nodes of the cluster and submits the topology to the master node, also called Nimbus. The worker nodes receive task assignments from the master and execute the tasks assigned to them. The coordination between the master node and the worker nodes is handled by nodes running Apache Zookeeper. Figure \ref{fig:storm_arch} shows a graphical representation of Storm architecture.

\subsection{Nimbus Node}

The master node runs a server daemon called Nimbus. The main role of Nimbus is to receive topology submissions from clients. Upon receiving a topology submission, Nimbus takes the following steps:

\begin{description}
	\item[Validate the topology] \hfill \\
	The topology is validated using a validator to ensure the submitted topology is valid before trying to execute it. Programmers using Storm can write their own validator by implementing the \texttt{ITopologyValidator} interface.
	\item[Distribute the topology source code] \hfill \\
	Nimbus ensures that the workers involved in the topology computation have the source code by sending it to all nodes of the cluster.
	\item[Schedule the topology] \hfill \\
	Nimbus runs a scheduler that distributes the work among workers of the cluster. Similarly to validation, the user can use his own scheduler by implementing the \texttt{IScheduler} interface or use the default scheduler provided by Storm. The default scheduler uses a simple Round-robin strategy \citep{Aniello_Baldoni_Querzoni_2013}.
	\item[Activate the topology] \hfill \\
	Nimbus transitions the topology to active state which tells the worker nodes to start executing it.
	\item[Monitor the topology] \hfill \\
	Nimbus continues to monitor the topology by reading heartbeats sent by the worker nodes to ensure that the topology is executing as expected and worker nodes have not failed.
\end{description}


Nimbus is an Apache Thrift \cite{ApacheThrift} service (more on Thrift in section \ref{sec:serialisation}) that listens to commands submitted by clients and modifies the state of a cluster accordingly. Following are the commands supported by Nimbus:

\begin{description}
	\item[Submit a topology] \hfill \\
	Clients can submit a topology defined in a Java Archive (JAR) file. The Nimbus service then ensures that the topology configuration and resources are distributed across the cluster and starts executing the topology as previously described.
	\item[Kill a topology] \hfill \\
	Nimbus can stop running a topology and remove it from the cluster. The cluster can still continue executing other topologies.
	\item[Activate/deactivate a topology] \hfill \\
	Topologies can be deactivated and reactivated by Nimbus. This could be useful if the spout temporarily cannot produce a stream and the user does not want the cluster to execute idly.
	\item[Rebalance a topology] \hfill \\
	Nimbus can rebalance a topology across more nodes. Thus if the number of nodes in the cluster ever changes the user can increase or decrease the number of nodes involved in the topology computation.
\end{description}

\subsection{Worker Nodes}

The worker nodes run a daemon called Supervisor. There are 4 layers of abstraction which control the parallelism of a worker node.

\begin{description}
	\item[Supervisor] \hfill \\
	A supervisor is a daemon process the user runs on a worker node to make it part of the cluster. It launches worker processes and assigns them a port they can receive messages on. Furthermore, it monitors the worker processes and restarts them if they fail. A worker node runs only one supervisor process.
	\item[Worker] \hfill \\
	A worker process is assigned a port and listens to tuple messages on a socket associated with the port. A worker launches executor threads as required by the topology. Whenever it receives a tuple, it puts it on a receive queue of the target executor.
	
	Furthermore, the worker has a transfer queue where its executors enqueue tuples ready to be sent downstream. There can be multiple worker processes running inside one supervisor.
	\item[Executor] \hfill \\
	An executor controls the parallelism within a worker process. Every executor runs in a separate thread. An executor's job is to pick up tuples from its receive queue, perform the task of a component it represents, and put the transformed tuples on the transfer queue of the worker. There can be many executors running inside one worker and an executor performs one (the usual case) or more tasks.
	\item[Task] \hfill \\
	A task represents the actual tuple processing function. However, within an executor thread all the tasks are executed sequentially. The main reason for having tasks is that the number of tasks stays the same throughout the lifetime of a topology but the number of executors can change (by rebalancing). Thus if some worker nodes in the cluster go down, the topology can continue executing with the same number of tasks as before.
\end{description}

\subsection{Zookeeper Nodes}
\label{subsec:zookeeper}

A Storm cluster contains a number of Zookeeper nodes which coordinate the communication between Nimbus and the worker nodes. Storm does this by storing the state of the cluster on the Zookeper nodes where both Nimbus and worker nodes can access it.

The cluster state contains worker assignments, information about topologies, and heartbeats sent by the worker nodes to be read by Nimbus. Apart from the cluster state, Storm is completely stateless. Hence, if the master node or a worker node fail the cluster continues executing and the node will get restarted if possible. The only time the cluster stops executing  completely is if all the Zookeper nodes die.

\section{Serialisation}
\label{sec:serialisation}

Since Storm topologies execute on a cluster all components need to be serialisable. This is achieved with Apache Thrift. Components are defined as Thrift objects and Thrift generates all the Java serialisation code automatically.

Furthermore, since Nimbus is a Thrift service Thrift generates all the code required for remote procedure call (RPC) support. This allows defining topologies in any of the languages supported by Thrift and easy cross-language communication with the Nimbus service.

\todo{multilang}

\todo{metrics}

\todo{hooks}


\chapter{Bringing Storm to Multi-core}

The following chapter explains the design of Storm-MC. We describe how Apache Storm behaves on multi-core machines (\ref{sec:storm_on_mc}), how the Storm architecture was ported over to multi-core (\ref{sec:storm_mc_arch}), and we list feature differences between Apache Storm and Storm-MC (\ref{sec:differences}).

\section{Apache Storm on Multi-core}
\label{sec:storm_on_mc}

To begin, we discuss why Apache Storm does not perform optimally on a single multi-core machine. Storm can be ran in local mode where it emulates execution on a cluster. This mode exists so that it is possible to debug and develop topologies without needing access to a cluster. However, there are several reasons why the local mode is not as performant as it could be.

\subsection{Tuple Processing Overhead}

\begin{figure}[!htb]
	\centering
	\includegraphics[scale=0.7]{pdf/worker_inside.pdf}
	\caption{Tuple processing in Apache Storm.}
	\label{fig:worker_inside}
\end{figure}

Figure \ref{fig:worker_inside} shows how tuple processing is implemented inside a Storm worker process. The tuple is read from a message buffer by the receiver thread of the worker and put on a receive queue of the target executor. The tuple is then picked up by the component thread of the executor for task execution.

After the component thread has executed the task it puts the tuple on the executor send queue. There, it is picked up by the executor sender thread which puts the tuple on the global send queue of the worker. Finally, the global sender thread of the worker serialises the tuple and sends it downstream.

Alternatively, if the tuple is forwarded to an executor in the same worker process it is put on the receive queue of the corresponding executor directly after task execution.

The queues used in Storm are implemented as ring buffers using the LMAX Disruptor library \citep{LMAXDisruptor}. Detailed background on how Disruptor works and its performance benchmarks can be found in \citep{Thompson_Farley_Barker_Gee_Stewart_2011}.

There is significant overhead required to simulate sending tuples to executors in other worker nodes. For one, there is the overhead from the tuple passing through the three queues of a worker. The authors of LMAX Disruptor showed that a three step pipeline has half the throughput of a single consumer-producer pipeline \citep{DisruptorWiki}.

Furthermore, to emulate over-the-network messages Storm uses a \texttt{Hashmap} of \texttt{LinkedBlockingQueue}s which according to \cite{Thompson_Farley_Barker_Gee_Stewart_2011} has several orders of magnitude lower performance than the Disruptor. Due to less write contention, lower concurrency overhead, and being more cache-friendly the Disruptor pattern can offer latency of inter-thread messages lower than 50 nanoseconds and a throughput of over 25 million messages per second.

\subsection{Thread Overhead}

\begin{description}
	\item[Acker Bolt] \hfill \\
	The Acker is included in every topology. The Acker bolt can be disabled via the configuration file. In such a case it is mostly idle since it does not receive any messages but it can still use up resources especially if it waits for messages using a busy waiting strategy.
	\item[Heartbeats \& Timers] \hfill \\
	Every worker has a heartbeat thread that simulates sending heartbeat messages to the Nimbus node. It does this by writing to a local cache which is persisted to a file by a write on every heartbeat. Since the write is implemented using the \texttt{java.io} package the write is blocking - the thread cannot continue until the write is completed. While heartbeats are essential in cluster mode to signal the node being alive, there is no need for them in local mode.
	\item[Zookeeper Emulation] \hfill \\
	More overhead is produced by a local Zookeper server which emulates the Zookeeper nodes of a cluster. Running the Zookeeper server is a massive addition to the list of overheads as shown in the following paragraphs. The purpose of Zookeeper is to maintain states of running topologies and nodes of the cluster. As we will show in the following sections maintaining this state on multi-core is not necessary.
\end{description}

During profiling we found that a topology with one worker and three executors was being executed with 55 threads (not including system JVM threads and threads created by the profiler). Table \ref{table:breakdown} shows a breakdown of what the individual threads were used for.

\begin{table}[htb!]
\centering
\small
\begin{tabular}{@{}ll@{}}
    \textbf{Spout Parallelism} & \textbf{\# of Threads} \\ \toprule
    Main Thread & 1  \\
	Worker Sender \& Receiver Threads & 2  \\
    Acker \& System Component Threads & 2  \\
    Executor Component Threads & 3  \\
    Executor Sender Threads & 5  \\
    Various Timers \& Event Loops & 14  \\
    Zookeper Server & 28  \\
\end{tabular}
\caption{Breakdown of threads used by Storm to execute a 3-component topology.}
\label{table:breakdown}
\end{table}

To find out what state the threads were actually in at any given time the topology was executed for three minutes and a JVM thread dump was recorded every second. The average results of this experiment can be observed in table \ref{table:dump} and the state distribution over time can be seen in Figure \ref{fig:dump-plot}.

\begin{table}[htb!]
\centering
\small
\begin{tabular}{@{}lc@{}}
    \textbf{Spout Parallelism} & \textbf{\# of Threads} \\ \toprule
    RUNNABLE & 8  \\
	TIMED WAITING & 22  \\
    WAITING & 25  \\
\end{tabular}
\caption{Average number of recorded thread states over a three minute period.}
\label{table:dump}
\end{table}

<<dump-plot, echo=FALSE, cache=TRUE, fig.cap="Thread state distribution over time.", fig.pos="!htb", fig.height=3>>=
@

Even though three minutes may seem to be a very short amount of time the fact that there is almost no variation shows that it is sufficient. As can be seen from the table, most of the threads were either in state \texttt{WAITING} or \texttt{TIMED WAITING}. According to the Java documentation on thread states \citep{JavaThreads} these two states are used for threads that are waiting for an action from a different thread and cannot be scheduled by the scheduler until that action is executed.

On average there were eight threads in state \texttt{RUNNABLE} which JVM uses to mark threads which are executing on the JVM and are possibly waiting for resources from the operating system (OS) such as processor \citep{JavaThreads}. Hence, these are threads directly competing to be scheduled by the OS. This means that for three components running in parallel there are five threads doing potentially unnecessary work.

In the subsequent sections we will show that these threads were in fact unnecessary and we will discuss how the number of threads was reduced. In fact, to execute the same topology on Storm-MC requires only 5 threads.

\section{Storm-MC Design}
\label{sec:storm_mc_arch}

The design we adopted for porting worker nodes is to only have one worker process running all the executor threads of a topology.

Additionally, the code for the Nimbus service was merged with the worker. This was done because there is no need to run Nimbus and worker specific code at the same time. Once Nimbus sets up the topology, all the work is done by the worker. Hence they can be executed serially.

\subsection{Nimbus}

Unlike Storm executing on a cluster, Storm-MC does not support running multiple topologies at the same time. However, to do that one only needs to run the topology in a separate process. This is because unlike when executing on the cluster different topologies do not need to share any state and it is more natural to execute them as separate processes.

This has the added benefit of each process having its own part of main memory thus reducing cache conflicts as shown in \citep{Chandra:2005:PIC:1042442.1043432} and providing higher security by not having different topologies share memory space. Additionally, if a single thread of one topology is blocking it does not block other topologies.

Storm-MC does not support topology scheduling. Since within one process there is always only one topology running at a time and the hardware configuration of the machine does not change, the parallelism is clearly defined by the number of executors per component specified in the topology configuration.

One way to implement scheduling could be to pin threads to specific cores. Unfortunately, Java does not provide support for CPU affinity, the assignments are handled automatically by the JVM. Potentially, this could be achieved by using C or C++, both of which support CPU affinity, but this was not implemented in Storm-MC.

The role of Nimbus in Storm-MC has effectively been reduced to validating the topology and passing it along to the worker source code which handles topology execution.

\subsection{Worker}

In Apache Storm, a worker node runs the supervisor daemon, which in turns launches worker processes which run executors which execute tasks. In Storm-MC, however, there is only one worker process which runs all the executors and their tasks. This design has several benefits:

\begin{itemize}
	\item All the inter-thread communication occurs within one worker process.
	\item The supervisor daemon can be removed as there is no need to synchronise or monitor workers.
	\item There is no need to simulate over-the-network message passing.
	\item Message passing between executor threads within a worker stays the same as in Apache Storm.
\end{itemize}

A comparison of an Apache Storm worker node and its Storm-MC equivalent is shown in Figure \ref{fig:comparison}.

\begin{figure}[!htb]
\centering
\begin{subfigure}{.5\textwidth}
  \centering
  \includegraphics[width=0.95\linewidth]{pdf/distributed_worker.pdf}
  \caption{Worker node in Apache Storm.}
  \label{fig:comparison1}
\end{subfigure}%
\begin{subfigure}{.5\textwidth}
  \centering
  \includegraphics[width=0.95\linewidth]{pdf/local_worker.pdf}
  \caption{Worker node equivalent in Storm-MC.}
  \label{fig:comparison2}
\end{subfigure}
\caption{Comparison of a worker in Storm and Storm-MC}
\label{fig:comparison}
\end{figure}

The role of the worker is to launch executors and provide them with a shared context through which they can communicate. This is done with a map of executor identifiers to Disruptor queues which the executors use to pass tuples between each other.

\todo{Write about disruptor here.}

Moreover, the worker contains a map of components to streams and fields. This map is used by components to figure out which components subscribe to streams they produce.

Additionally, a worker has a user timer which components can use to get tick tuples at regular intervals.

Finally, a worker contains shared resources which can be used by the executors. One such resource is a configurable-size thread-pool \texttt{ExecutorService}. Executors can use these via the topology context to execute some computation on a separate thread.

\todo{Refactor below section}

\subsection{State}

As mentioned before, Storm-MC is completely stateless. The cluster state that was managed by Zookeeper in Apache Storm was completely stripped away. This state was only relevant when multiple topologies were sharing resources.

\subsection{Serialisation}

Great amount of work was put into removing the dependency of Storm-MC on Apache Thrift. This was mostly done to reduce code bloat and remove an unnecessary dependency since there is no serialisation required in a multi-core environment.

This required refactoring all the data types generated automatically by Thrift. This also significantly reduced the size of the codebase and made the code more readable and self-documenting than the code generated by Thrift.

\section{Implementation Details}

\begin{figure}[!htb]
	\centering
	\includegraphics[scale=0.7]{pdf/worker_inside_mc.pdf}
	\caption{Tuple processing in Storm-MC.}
	\label{fig:worker_inside_mc}
\end{figure}

The implementation of tuple processing in Storm-MC is depicted in Figure \ref{fig:worker_inside_mc}. As can be seen from the figure, the queues used for remote message sending were stripped away and there is only one Disruptor queue for every executor. Once an executor is done processing a tuple it simply puts it on the Disruptor queue of its downstream bolts.

Thus the tuple processing in Storm-MC is a variant of multiple producer single consumer problem. We considered several other options such as \texttt{ArrayBlockingQueue} when implementing the tuple processing mechanism. However, the Disruptor shows superior throughput and latency compared to alternative solutions \citep{DisruptorWiki}.

\todo{Storm-MC requires one thread per component, user timer and main thread}

\subsection{Nimbus}

\subsection{Worker}

As mentioned in previous chapter Storm-MC was implemented with only one worker per topology. This worker then exposes a common context to all the executors.

\subsection{Executor Receive Queues}

Worker stores a map of executor ids to receive queues. These are then used by executors to pass tuples between each other. As discussed in the previous chapter, the queues are implemented as Disruptor ring buffers. The way this works is as follows:

The target executor waits at for the next entry of the ring buffer to become available. There are four different waiting strategies an executor can employ:

\begin{description}
	\item[BlockingWaitStrategy] \hfill \\
	This strategy uses a lock and a condition variable. 
	\item[YieldingWaitStrategy] \hfill \\
	This strategy initially spins for hundred iterations and then uses \texttt{Thread.yield()}.
	\item[SleepingWaitStrategy] \hfill \\
	This strategy initially spins for hundred iterations, then uses \texttt{Thread.yield()}, and finally uses \texttt{LockSupport.parkNanos(1L)} to sleep.
	\item[BusySpinWaitStrategy] \hfill \\
	In this strategy the thread is in a so-called tight loop, where it checks whether a new entry is available, breaks out if it is. Otherwise it continues looping.
\end{description}

\subsection{Executor}

An executor is implemented as a single thread of execution. This thread consumes a batch of tuples from the Disruptor queue when it becomes available, performs the tasks of a component it represents, and finally places the newly produced tuples on the Disruptor queue of other executors.

In Apache Storm, an executor runs two threads in parallel. One that picks up new tuples from the receive queue

\todo{Describe how executor is implemented - Thread vs Executor}

\section{Differences between Apache Storm and Storm-MC}
\label{sec:differences}

The codebase of Apache Storm is fairly large - 54,985 lines of code as reported by \texttt{cloc} \citep{Cloc}. Thus we had to prioritise features that were ported over to Storm-MC. Table \ref{table:features} presents a list of Storm features and shows which were ported over to Storm-MC and which were not.

\begin{table}[h!]
\centering
\small
\begin{tabular}{@{}lcc@{}}
    \textbf{Feature} & \textbf{Apache Storm} & \textbf{Storm-MC} \\ \toprule
    Multi-language Topologies & \cmark & \cmark \\
    Hooks & \cmark & \cmark \\
    Metrics & \cmark & \cmark \\
    Tick Tuples & \cmark & \cmark \\
    Multiple Topologies & \cmark & \xmark \\
	Trident API & \cmark & \xmark \\
    System Bolt & \cmark & \xmark \\
    Built-in Metrics & \cmark & \xmark \\
    Nimbus as a Server & \cmark & \xmark \\
\end{tabular}
\caption{Feature comparison of Apache Storm and Storm-MC.}
\label{table:features}
\end{table}


\chapter{Evaluation}

In this chapter we evaluate Storm-MC. We do this by comparing its performance against the local mode of Apache Storm.

\section{Performance}

\subsection{Software Setup}

\todo{Change versions below as applicable. Link to GitHub for source?}

All performance benchmarks were ran using the following software packages:

\begin{itemize}
	\item Apache Storm version 0.9.2
	\item Storm-MC version 0.1.5
	\item A fork of IBM Storm Email Benchmarks version 0.1.4
	\item Storm-benchmark version 0.1.0
\end{itemize}

The Apache Storm source code had to be adapted to include a workaround for a deadlock bug present in version 0.9.2. This bug caused a topology to exit with threads left in Zombie state under certain conditions. This prevented Storm from logging the benchmark metrics after execution.

Version 0.1.5 is the latest version of Storm-MC as of this moment. The first release was version 0.1.0 which was production-ready but since then there were 5 minor versions fixing bugs as they were discovered during testing.

IBM open sourced a suite of benchmarks which they used to compare Apache Storm to their real-time stream system IBM Infosphere Streams. These benchmarks were adapted and used to benchmark Apache Storm and Storm-MC.

Lastly, a number of spout and bolt components were used from the storm-benchmark project which Apache Storm developers use to benchmark Storm.

Since Storm-MC reuses package names from Apache Storm, the same benchmark is directly executable by both libraries. This saved a lot of time and furthermore there is no need to maintain two benchmarks suites.

\todo{Go into more detail which components were re-used and where?}

\subsection{Hardware Setup}

The benchmarks were executed on the following hardware:

\todo{This is currently student.compute, find out what it is.}

\subsection{WordCount Topology}

The first topology we tested for performance is a variant of the aforementioned WordCount topology. Recall, that this topology is shown graphically on figure \ref{fig:wordcount_topology}. The topology was ran with 3 executors running for every component.

This topology is considered to be CPU-intensive.



\begin{knitrout}
\definecolor{shadecolor}{rgb}{0.969, 0.969, 0.969}\color{fgcolor}\begin{figure}

{\centering \includegraphics[width=\maxwidth]{figure/wordcount-plot-1} 

}

\caption[Improvement of Storm-MC over Apache Storm in number of tuples processed]{Improvement of Storm-MC over Apache Storm in number of tuples processed}\label{fig:wordcount-plot}
\end{figure}


\end{knitrout}


\subsection{Enron Topology}

Next, we tested the Enron topology from the IBM benchmarks. In this topology, serialised emails from the Enron email database are read from a file by a spout. They are further deserialised by one bolt, filtered by another bolt, modified by yet another bolt and then finally metrics are recorded by another bolt.

Similarly, to the WordCount topology this topology is serial in nature. However, whereas the WordCount topology keeps the random sentences in memory, the Enron topology reads from a file. Thus, this benchmark is mostly I/O intensive.

\subsection{RollingSort Topology}

The RollingSort topology is ported over from the aforementioned storm-benchmark project. This topology includes one spout and one bolt. The spout produces hundred character-long strings of random digits from zero to eight. The bolt stores a rolling window of hundred of these messages and sorts them every \textbf{x} seconds.

This benchmark is included because it is considered memory-intensive.

\todo{change x depending on the actual benchmark.}

\section{Challenges}

In this section we are going to discuss challenges we encountered while porting Apache Storm to multi-core machines.

\chapter{Conclusion}

This final chapter concludes with a discussion on future work that could stem from this project (\ref{sec:contribs}), describes challenges encountered while building Storm-MC (\ref{sec:challenges}), and presents a summary of contributions of this project (\ref{sec:contribs}).

\section{Future Work}
\label{sec:future_work}

Storm-MC could be improved in a number of ways. Following are ideas that were out of scope of this project but we would like to see get implemented in the future:

\begin{description}
	\item[Storm-MC as a Server] \hfill \\
	Storm-MC could be updated to allow server-like execution. This could have several benefits such as being able to execute multiple topologies at the same time with a thin wrapper that could control their execution just like the Nimbus service in Apache Storm. This was not implemented as part of this project as we assumed most of the time users would be executing one topology at a time.
	\item[Higher Level Abstractions] \hfill \\
	Defining components of a Storm-MC topology is fairly simple. Users of the library only need to define how components are connected and how they consume and produce tuples. However, this could be taken even further with the user specifying high-level functions and the Storm-MC library figuring out how to parallelise the work that needs to be carried out. In Apache Storm this is implemented with the Trident API which was not ported as part of this project.
	\item[Automatic Parallelism] \hfill \\
	Sometimes when configuring a topology it may be difficult to predict the rate at which spouts are going to produce tuples. If the rate is underestimated consumers could be lagging behind producers. On the other hand, if the rate is overestimated consumers could be idle, not doing any useful work. Thus it could be advantageous to have an automatic parallelism setting which could add or remove consumers based on the current tuple rate.
	
	It may seem that this would be trivial to implement with a pool of threads representing one component. However, there are several problems that need to be considered. For example, fields grouping guarantees that tuples with the same field values go to the same executor. Changing the parallelism at runtime breaks this guarantee.
	
	Alternatively each executor could use a pool of threads. This comes with its own set of problems: the executor object would have to provide synchronised access to the pool which would only increase tuple latency.
	\item[Performance Comparison with Distributed Storm] \hfill \\
	The benchmarks in this report compared Storm-MC to Apache Storm running in local mode. It would be interesting to see how Storm-MC compares to Apache Storm running on a cluster. One could compare the number of nodes required in a cluster to the number of cores required in a multi-core server to achieve certain throughput for a given topology. Benchmarks like this could provide insight into when it becomes advantageous to deploy the topology to a cluster. These benchmarks were not included in this project because we did not have access to a cluster.
\end{description}

\section{Challenges}
\label{sec:challenges}

In this section we discuss challenges we encountered while porting Apache Storm to multi-core. We also try to provide a critical analysis of the project.

\begin{description}
	\item[Storm API Compatibility] \hfill \\
	A big challenge while working on this project was ensuring that the final system is backwards compatible with the Apache Storm API. Doing this ensured that existing applications developed for Apache Storm can be executed with Storm-MC. On the other hand, this was sometimes limiting the possible performance improvements. An example of this is the automatic parallelism scaling mentioned in the previous section.
	\item[Unfamiliarity with Clojure] \hfill \\
	One of the main challenges while working on this project was learning a new programming language - Clojure. Since most of the implementation of Apache Storm is written in Clojure, this language had to be studied and its concepts well understood for us to be able to write code that worked with the existing codebase. By the end of the project writing Clojure has become second nature to us but initially progress was slow.
	\item[Lack of Documentation] \hfill \\
	Even though Apache Storm is a popular project documentation is available only for the high level concepts used within Storm. The implementation details are often obscured away in hard to understand functions. Since the documentation is lacking our knowledge of Storm had to be obtained by reading the source code of an initially unfamiliar language. By the end of the project the Storm-MC codebase became well documented and we might attempt back-porting parts of it to Apache Storm.
\end{description}

\section{Summary of Contributions}
\label{sec:contribs}

The primary contribution of this project is Storm-MC - a library aimed at data stream processing applications. The benefits of using Storm-MC are twofold:

\begin{itemize}
	\item It offers the same easy-to-use API as Apache Storm.
	\item It is tailored to multi-core environments.
\end{itemize}

Since Storm-MC uses the same API as Apache Storm, applications written with Storm in mind can be ported to use Storm-MC with minimum amount of effort. Thus if an application requires parallelism satisfiable by a single multi-core machine, it can be executed efficiently on one machine instead of a cluster.

Moreover, the Storm API allows programmers to create data stream processing applications on multi-core with an unprecedented ease. All of this comes with the superior performance Storm-MC offers compared to running Apache Storm in local mode, as shown in Section \ref{sec:performance}.

\begin{appendices}
\chapter{Listings}
\label{ch:listings}

\begin{listing}[!htb]
\inputminted{java}{code/RandomSentenceSpout.java}
\caption{RandomSentenceSpout.java}
\label{listing:wordcount_spout}
\end{listing}

Listing \ref{listing:wordcount_spout} shows the definition of a spout that emits a randomly chosen sentence from a predefined collection of sentences.

\begin{listing}[!htb]
\inputminted{java}{code/SplitSentence.java}
\caption{SplitSentence.java}
\label{listing:wordcount_split}
\end{listing}

\begin{listing}[!htb]
\inputminted{python}{code/splitsentence.py}
\caption{splitsentence.py}
\label{listing:wordcount_split_py}
\end{listing}

Listings \ref{listing:wordcount_split} and \ref{listing:wordcount_split_py} show how a bolt defined in Python can be part of this Java-defined topology.

\begin{listing}[!htb]
\inputminted{java}{code/WordCount.java}
\caption{WordCount.java}
\label{listing:wordcount_count}
\end{listing}

Finally, listing \ref{listing:wordcount_count} shows how a bolt that counts the number of word occurrences can be implemented.


\end{appendices}

% use the following and \cite{} as above if you use BibTeX
% otherwise generate bibtem entries
\bibliographystyle{apalike}
\bibliography{mybibfile}

\end{document}
